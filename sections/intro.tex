\section{Introduction}

Motion blur is a challenging problem in computer vision.
It occurs when moving objects are captured with long exposures.
This leads to a loss of information in images, and thus algorithms for deblurring are pursued.
Deblurring motion blur due to fast moving objects~\citep{fmo} (FMOs) is a challenging subset of this task, and it is necessary for tracking FMOs such as in ball-based sports, free-falling objects, high-speed cars, etc.

% TODO:
% * Cite all other deblurring papers
% * Extend into proper literature review
Various deep learning approaches have been proposed to tackle FMO blur.
\citet{defmo} proposed an algorithm known as DeFMO, that uses a convolutional encoder-decoder architecture to deblur individual images by generating sub-frames that depict the motion of the FMO.\@
The model is trained in a supervised learning setting with a wide variety of losses for good performance.

While these losses lead to good performance on various datasets, they might not be optimal for the best performance.
Here, we aim to replace these hand-engineered losses with neural networks that can learn a loss function for training the base model.
For this, we use Generative Adversarial Networks~\citep{gan} (GANs).

The \textbf{contributions} of this project are:
\begin{itemize}
    \item Improvement of the DeFMO model's outputs using GANs.
    \item A novel learned loss function for temporal frame consistency.
    \item A generalized approach over DeFMO for any task that involves video super-resolution.
    \item Optimizations for more efficient memory management during training.
    \item Refactoring and code quality enforcement over the DeFMO codebase.
\end{itemize}
